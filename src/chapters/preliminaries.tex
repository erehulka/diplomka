\chapter{Preliminaries}\label{ch:preliminaries}

Definitions not provided in our work can be found in the book \quotes{Graph Theory} by Diestel \cite{Diestel}. All graphs considered in this work are undirected, connected and finite. We do not allow trivial graphs, but sometimes for the sake of complete definition we address them. We permit multiple edges, however loops are not allowed.

While we work only with cyclic connectivity, for the sake of providing the definition completely in our work, we shall start with definining how we will compute connectivity, but more importantly, how we will work with edge cuts.

\section{Multipoles}\label{sec:multipoles}

In our work, we allow a specific type of graph, which allows ends of its edges to not be incident with a vertex, resulting in a graph with dangling, or even so-called isolated edges. Such structures are called \textit{multipoles}. This term was first used by Fiol in 1991 \cite{Fiol1991}, however we will be following a definition by Nedela and Škoviera \cite{Nedela1996}.

\begin{definition}[\cite{Nedela1996}]
	A \textit{multipole} is a pair $M=(V,E)$ of distinct finite sets of vertices $V$ and edges $E$, where every edge $e\in E$ has two edge ends, which may or need not be incident with a vertex.
	
	According to the incidence of edge ends, we define four types of edges:
	\begin{enumerate}[nolistsep]
		\item A \textit{link} is an edge whose ends are incident with two distinct vertices.
		\item A \textit{dangling edge} is an edge whose one end is incident with a vertex, and the other is not.
		\item An \textit{isolated edge} is an edge whose both ends are not incident with any vertex.
	\end{enumerate}
\end{definition}

\todo{dokonci nejake definicie, ako napriklad semiedges, junction a podobne}

When we write some link $e$ as $xy$, we mean a link between vertices $x$ and $y$. On the other hand, when we just regard it as an edge $xy$, then $x$ and $y$ are its edge ends.

Since we allow these dangling and isolated edges, we can introduce so-called \textit{edge severing}. By severing an edge $e=e_1e_2$, we mean the removal of $e$, along with adding two new edges: $e_1e_3$ and $e_4e_2$, where $e_3$ and $e_4$ are new edge ends. It is evident that by severing a link we obtain two new dangling edges, by severing a dangling edge we obtain a new dangling edge and an isolated edge, and by severing an isolated edge we obtain two new isolated edges.

\todo{Celkom chudobná kapitola, bolo by fajn doplniť ešte zvyšok definícíi}

\section{Edge Cuts and Connectivity}\label{sec:edge-cuts}

\todo{Daj si pozor, ci nebude problem ze to definujes takto. Lebo v cyklickych k-castiach sa spomina ze 2 komponenty len, nie ze disconnected.}

As the name suggests, an \textit{edge cut} decomposes a graph into multiple connected components. It is a set of edges of a graph $G$, such that if we sever every edge in the set, the resulting graph will not be connected. There is also a second way to denote an edge cut, which is helpful when trying to determine how will the vertices be split after severing the edges. Note that by $E(V_1,V_2)$ we denote the set of edges, which have one end in $V_1$ and the other in $V_2$.

Let $G$ be a graph and ${V_1,V_2}$ a partition of $V(G)$, such that both sets are non-empty. By $E(V_1,V_2)$ we denote the set of edges, which have one end in $V_1$ and the other in $V_2$.

Note, that $E(V_1,V_2)$ is indeed an edge cut, when we sever all of these edges, there will not be any edge between a vertex in $V_1$ and $V_2$, splitting the graph into two components. The sets $V_1$ and $V_2$ are called the sides of the cut.

A graph which results from a graph $G$ by severing edges from a set $S$ is denoted as $G-S$. Note that this does not require $S$ to be an edge cut, the result can still be connected.

A graph $G$ is \textit{k-edge-connected} (for $k\in\mathbb{N}$), if $|G|>k$ and $G-S$ is connected for every set $S\subseteq E(G)$ such that $|S|<k$. In other words, there is no edge cut with less than $k$ edges.

\todo{nejaky textik}

\begin{definition}
	Let $G$ be a graph. The \textit{edge conectivity} of $G$, denoted by $\lambda(G)$ is the greatest $k\in\mathbb{N}$ such that $G$ is $k$-edge-connected.
\end{definition}

In other words, it is the size of the smallest edge cut in the graph. It can be easily proven, that for each non-trivial graph $G:~\lambda(G)\leq\delta(G)$, where $\delta(G)$ is the minimum degree. The proof comes from the fact, that all edges incident with a vertex are indeed edge cuts, separating the vertex from the rest of the graph, meaning there is an edge cut of size $\delta(G)$ in each graph. 

\section{Cyclic Edge Connectivity}\label{sec:cyclic-edge-connectivity}

The latter can be a huge limitation, since for example when trying to explore cubic graphs, there is already an upper bound on the edge connectivity. Because of this, a new metrics regarding edge cuts were introduced, requiring both of the components resulting after severing edges from an edge cut to contain cycles. The notation is similar as before, just with a keyword \textit{cyclic} before it. To maintain a reasonable level of formalism, we shall define these properties more precise.

An edge cut $E(V_1,V_2)$ in a graph $G$ is called \textit{cycle separating}, if both induced subgraphs $G[V_1]$ and $G[V_2]$ contain a cycle. In contrast to edge cuts, there are non-trivial graphs which do not contain a cycle separating edge cut \cite{atoms-of-cyclic, Lou2008, lovasz1965graphs}, for example trees. 

Similarly to edge connectivity, we say that graph $G$ is \textit{cyclically k-edge-connected} (for $k\in\mathbb{N}$), if there is no cycle separating edge cut with less than $k$ edges.

\begin{definition}[\cite{atoms-of-cyclic}]
	Let $G$ be a graph. The \textit{cyclic edge conectivity} of $G$, denoted by $\zeta(G)$ is the greatest $k\in\mathbb{N}$ such that $G$ is cyclically $k$-edge-connected.
\end{definition}

If there is no cycle separating edge cut in a graph $G$, we say that $\zeta(G)=\infty$ \cite{Lou2008}.


\todo{Síce som spomenul takmer všetko čo som chcel, ale je tu ešte veľa miesta. Určite sa bude dať niečo spomenúť.}
