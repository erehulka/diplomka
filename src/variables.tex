% Use values that are applicable to you and your thesis.

\newif\ifenglish{}
\englishtrue{} % english language
% \englishfalse{} % slovak language

\newif\ifbachelor{}
% \bachelortrue{} % bachelor's thesis
\bachelorfalse{} % master's thesis

\newif\ifcompsci{}
\compscitrue{} % computer science
% \compscifalse{} % bioinformatics

\newif\ifconsultant{}
% \consultanttrue{} % I have a consultant
\consultantfalse{} % no consultant

% The year in which you plan to turn the thesis in.
\newcommand{\thesisyear}{2025}

% Tip - use the `\\` to split a long thesis name nicely.
\newcommand{\thesisname}{Cyclic edge-connectivity of cubic graphs}

% If you are writing a master's thesis, don't forget to add Bc. here.
\newcommand{\thesisauthor}{Bc. Erik~Řehulka}

\newcommand{\thesissupervisor}{Mgr.~Jozef~Rajník,~PhD.}

\ifconsultant{}
    \newcommand{\thesisconsultant}{doc.~RNDr.~Cyril~Fiktívny,~PhD.}
\fi

\newcommand{\thesisacknowledgments}{
    I am deeply grateful to my supervisor, Jozef Rajník, whose guidance has been essential to this work. His insightful comments, direction toward meaningful results, and careful proofreading have been instrumental in completing this research. He also provided valuable guidance on English mathematical writing conventions, which helped improve the clarity of this thesis.
    
    I also extend my thanks to the participants of the Bratislava graph seminar, where I had the opportunity to present our results and receive valuable feedback. This platform proved especially beneficial, allowing us to identify and correct errors well before the submission deadline for this thesis.
    
    I am also grateful to the organizers of the Student Science Conference for providing an additional platform to present our results.
    
    Finally, I would like to thank the Faculty of Mathematics, Physics and Informatics for their support of this research.
}

\newcommand{\thesisabstractsk}{
    Cyklická hranová súvislosť je zaujímavá vlastnosť kubických grafov, ktorá sa ukázala ako významná v teórii grafov. Výskum ukázal, že vysoká cyklická hranová súvislosť koreluje s určitými štrukturálnymi vlastnosťami kubických grafov a minimálne možné protipríklady k viacerým otvoreným problémom musia mať vysokú cyklickú hranovú súvislosť. Teoretických nástrojov na určovanie cyklickej hranovej súvislosti však nie je veľa. V tejto práci skúmame kubické grafy zostrojené z menších stavebných blokov a ukazujeme ako môžeme určiť cyklickú hranovú súvislosť výsledného grafu pomocou stanovenia určitých podmienok pre tieto bloky. Naše teoretické výsledky ilustrujeme na praktických príkladoch.
}

\newcommand{\thesisabstracten}{
    Cyclic edge-connectivity is an interesting property of cubic graphs that has proven to be significant in graph theory. Research has shown that high cyclic edge-connectivity correlates with certain structural properties of cubic graphs, and notably, minimal potential counterexamples to several open problems need to have high cyclic \mbox{edge-connectivity}. However, the theoretical tools for determining cyclic edge-connectivity remain limited. In this work, we study cubic graphs built from smaller building blocks and demonstrate how we can determine the cyclic edge-connectivity of the resulting graph by imposing certain conditions for these blocks. We also illustrate our theoretical results through practical examples.
}

\newcommand{\thesiskeywordssk}{kubické grafy, cyklická hranová súvislosť, multipóly, inflácie}

\newcommand{\thesiskeywordsen}{cubic graphs, cyclic edge-connectivity, multipoles, inflations}

\newcommand{\thesischapters}{
    \input chapters/00-introduction.tex
    \input chapters/preliminaries.tex
    \input chapters/cyclic-edge-connectivity.tex
    \input chapters/algorithms.tex
    \input chapters/problem-formulation.tex
    \input chapters/results.tex
}

%\newcommand{\thesisappendices}{
%    \input chapters/91-appendixA.tex
%    \input chapters/92-appendixB.tex
%}

% Thesis type, field, programme and department are set here automatically, but please check that
% the values are indeed correct for you.
% If your supervisor works at FMFI, set the \thesisdepartment to the supervisor's department (this
% one should also be present in the thesis assignment in AIS). Otherwise, keep the Department of
% Computer Science.
\ifenglish{}
    \ifbachelor{}
        \newcommand{\thesistype}{Bachelor's Thesis}
    \else
        \newcommand{\thesistype}{Master's Thesis}
    \fi
    \ifcompsci{}
        \newcommand{\thesisfield}{Computer Science}
        \newcommand{\thesisprogramme}{Computer Science}
    \else
        \newcommand{\thesisfield}{Computer Science and Biology}
        \newcommand{\thesisprogramme}{Bioinformatics}
    \fi
    \newcommand{\thesisdepartment}{Department of Computer Science}
    \newcommand{\thesisfaculty}{Faculty of Mathematics, Physics and Informatics}
    \newcommand{\thesisuniversity}{Comenius University in Bratislava}
\else
    \ifbachelor{}
        \newcommand{\thesistype}{Bakalárska práca}
    \else
        \newcommand{\thesistype}{Diplomová práca}
    \fi
    \ifcompsci{}
        \newcommand{\thesisfield}{Informatika}
        \newcommand{\thesisprogramme}{Informatika}
    \else
        \newcommand{\thesisfield}{Informatika a Biológia}
        \newcommand{\thesisprogramme}{Bioinformatika}
    \fi
    \newcommand{\thesisdepartment}{Katedra informatiky}
    \newcommand{\thesisfaculty}{Fakulta matematiky, fyziky a informatiky}
    \newcommand{\thesisuniversity}{Univerzita Komenského v Bratislave}
\fi
\newcommand{\thesislocation}{Bratislava}

% Helpers

\newif\ifshowframe{}
% \showframetrue{} % show page frames to debug positioning
\showframefalse{} % normal
